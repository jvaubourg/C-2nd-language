\documentclass[10pt]{article}\usepackage[nu]{esial}
\CSH\sloppy
\usepackage{subfig,multirow}

\usepackage[utf8]{inputenc}
\hypersetup{colorlinks=false,pdfborder={0 0 0}}
\begin{document}
\title{TP4: Organisation de la mémoire, Makefiles et tableaux}
\maketitle

\begin{Exercice}\textbf{Organisation mémoire d'un processus}

On souhaite explorer la disposition mémoire d'un processus C, c'est-à-dire les
adresses mémoire où sont placés les différents éléments possible. Pour afficher
l'adresse à laquelle se trouve une variable \texttt{i} quelconque, il faut
utiliser: \run{printf("variable i: \%p",\&i);}

\Question Dans un fichier appelé \file{exo1.c}, déclarer les variables
suivantes :

\noindent\begin{minipage}{.55\linewidth} 
  \begin{enumerate}
  \item[(1)] une chaîne de 10 caractères globale et statique ;
  \item[(3)] un tableau de 10 entiers global ;
  \end{enumerate}   
\end{minipage}\begin{minipage}{.45\linewidth}
  \begin{enumerate}
  \item[(2)] un double local au main;
  \item[(4)] un caractère local statique ;
  \end{enumerate}
\end{minipage}
\begin{enumerate}
\item[(5)] un entier nommé \texttt{ex} déclaré en externe et sans valeur
  initiale.
\end{enumerate}

Dans la fonction main(), également dans \file{exo1.c}, faites écrire les
adresses des variables. Dans un deuxième fichier appelé exo1bis.c mettre la
définition de l'entier \texttt{ex} que vous initialiserez à la valeur 20.

\Question Compilez ce programme avec \run{gcc -c exo1.c} \run{gcc -c exo1bis.c}
puis réalisez l'édition de liens avec \run{gcc -o exo1 exo1.o exo1bis.o}.
Exécutez \cmd{exo1} et concluez sur la place des différentes variables. Quelle
est l'effet du modificateur \texttt{static}? Qu'en est-il de \texttt{extern}?

\Question Modifiez votre programme pour qu'il affiche l'adresse de la fonction
main() elle-même. Obtenir l'adresse d'une fonction est similaire à ce qu'on
fait pour une variable: \run{printf("\%p$\backslash$n",\&main);} Où est
placé le code de cette fonction?

\Question Ajoutez une variable de type \texttt{char*} dont le contenu est
l'adresse d'un bloc alloué dynamiquement avec l'appel \run{malloc(512);}. Où
est placée cette variable? Où est placé le bloc alloué?
\end{Exercice}

\Exercice\textbf{Compilation séparée et Makefile.}
Récupérez le code fourni sur la page web du cours
(\url{http://www.loria.fr/~quinson/Teaching/CSH/}).

\Question Vérifiez que les commandes \run{gcc -c code1.c} et \run{gcc -c
  code2.c} ne produisent pas de message d'erreur et que les fichiers
\file{code1.o} et \file{code2.o} ont été créés.

\Question À partir de ces deux fichiers et de \file{codage.c}, créez le fichier
exécutable \file{codage} par la commande \run{gcc -o codage codage.c code1.o
  code2,o}

\Question Exécutez-le, puis supprimez les fichiers \file{code1.o},
\file{code2.o} et \file{codage}.

%\medskip\noindent$\blacktriangleright$ \textit{Utilisation d'un Makefile.}

\Question Écrivez un fichier Makefile permettant de compiler ce projet. On
rappelle qu'un tel fichier est composé de différentes règles, chacune suivant
la syntaxe suivante (avec un caractère tabulation et non des espaces au début
des lignes de commande):
\begin{Verbatim}
<fichier cible> : <liste des dependances>
	<commande pour fabriquer la cible a partir des dependances>  
\end{Verbatim}

\Question Vérifiez que votre makefile exécute les bonnes actions dans les cas
suivants:
\begin{itemize}
\item Aucun binaire n'existe, tout est à reconstruire;
\item Effacement de l'un des \texttt{.o};
\item Effacement du binaire;
\item Modification de l'un des fichiers source;
\end{itemize}


\begin{Exercice}\textbf{Dice Poker}
  Ce jeu est une variation du Poker qui utilise des dés au lieu de cartes. Vous
  allez maintenant implémenter un programme permettant de jouer à ce jeu contre
  l'ordinateur.

  Le jeu consiste en plusieurs manches. Dans chaque manche, l'ordinateur et
  l'humain lancent 5 dés. Certaines combinaisons de valeur de dés permettent de
  l'emporter sur d'autres. La liste suivante énumère les combinaisons de la
  plus ``puissante'' à la moins ``puissante'':
  \begin{tabular}{|c|c|l|}\hline
    \textbf{Id}&\textbf{Nom}&\textbf{Description}\\\hline
    7&Full& Les cinq dés ont la même valeur\\\hline
    6&Quatre d'un coup& Quatre dés sur cinq ont la même valeur, et le dernier
                        est différent\\\hline
    5&House&Trois dés ont la même valeur, et les deux
            autre ont la même autre valeur\\\hline
    4&Straight& Cinq valeurs consécutives (même dans le
                désordre)\\\hline
    3&Trois d'un coup&Trois dés ont la même valeur, tandis que les
                      deux autres sont différents\\\hline
    2&Deux paires&Deux paires de dés ont la même valeur\\\hline
    1&Paire&Deux dés ont la même valeur, et les autres sont
            différents\\\hline
    0&Rien&Toutes les autres situations\\\hline
  \end{tabular}
  
  \Question Observez le code du programme \file{dice\_poker.c} fourni.
  Consultez la page man des fonctions utilisées que vous ne connaissez
  pas. Compilez-le et exécutez-le plusieurs fois pour essayer.

  \noindent\begin{minipage}{.77\linewidth}
    \Question Ajoutez une fonction \run{int manche()} à ce programme, qui crée
    deux tableaux de 5 entiers tirés aléatoirement, et les affiche sous la
    forme suivante:
  \end{minipage}\hfill\begin{minipage}{.19\linewidth}
    \begin{Verbatim}
Moi : 3 5 4 4 5
Vous: 3 3 6 1 2
    \end{Verbatim}
  \end{minipage}
  Vous utiliserez une fonction \run{void affiche(char *name, int hand[5])} pour
  afficher chaque ligne.

  \Question Réalisez une fonction \run{int identifie(int hand[5])} retournant
  l'identificateur du type de main obtenu (voir la colonne Id ci-dessus).

  Pour cela, il faudra calculer la fréquence de chaque valeur de la main (dans
  un tableau temporaire). Ainsi, pour la main \{3,5,4,4,5\} la table de
  fréquence est \{0,0,1,2,2,0\} (il y a zéro fois la valeur 1, zéro fois la
  valeur 2, une fois la valeur 3, et ainsi de suite). Le cas du ``Straight'' se
  traite à part.

  \Question Modifiez la fonction \texttt{affiche()} pour qu'elle affiche le
  type de main obtenu par chacun.

  \noindent\begin{minipage}{.62\linewidth}

    \Question Modifiez \texttt{manche()} pour qu'elle affiche le gagnant de la
    manche et renvoi -1 si l'ordinateur gagne, 0 sur match nul et 1 si l'humain
    gagne. 

  \end{minipage}\hfill\begin{minipage}{.35\linewidth}
    \begin{Verbatim}
Moi : 3 5 4 4 5 (deux paires)
Vous: 3 3 6 1 2 (paire)
JE GAGNE !
    \end{Verbatim}
  \end{minipage}

  \Question Modifiez la fonction principale pour jouer 5 manches, compter
  les points acquis à chaque manche et afficher le résultat final.

  \Question Modifiez votre programme pour autoriser les jeux avec un nombre
  arbitraire de dés, spécifié en ligne de commande. La fonction d'évaluation
  doit naturellement être modifiée (on peut laisser les Straight de coté dans
  un premier temps).  Il faut pour cela consulter non seulement la table des
  fréquences de valeur, mais également la table des fréquences de la table des
  fréquences. 

  Par exemple, \{4,6,2,4,3,6,6,6,3\} a la table de fréquence suivante:
  \{0,1,2,2,0,5\} et la table de fréquence des fréquences suivante:
  \{2,1,2,0,0,1,0,0,0,0\}. Il y a deux valeurs qui ne sont pas représentées 
  (le 1 et le 5), une valeur unique (le 2), deux paires (3 et 4), et
  un groupe de cinq valeurs identiques (les 6). 


  \newcommand{\RR}[1]{\multirow{2}{*}{#1}}\vspace{-\baselineskip}
  \begin{figure}[h]
    \centering
    \subfloat[Table des fréquences]{
      \begin{tabular}{|l|c|c|c|c|c|c|}\hline
        Valeur    & 1 & 2 & 3 & 4 & 5 & 6 \\\hline
        Fréquence & 0 & 1 & 2 & 2 & 0 & 5 \\\hline
      \end{tabular}
    }
%
    \subfloat[Table des fréquences de fréquence]{
      \begin{tabular}{|l|c|c|c|c|c|c|c|c|c|c|}\hline
        Fréquence             & 0 & 1 & 2 & 3 & 4 & 5 & 6 & 7 & 8 & 9 \\\hline
        Nombre de valeurs     &\RR2&\RR1&\RR2&\RR0&\RR0&\RR1&\RR0&\RR0&\RR0&\RR0\\
        ayant cette fréquence &   &   &   &   &   &   &   &   &   &   \\\hline
      \end{tabular}
    }
  \end{figure}
  
  Un \textit{full} d'une main de 9 dés a la table de fréquence des fréquences
  suivantes: \{5,0,0,0,0,0,0,0,0,1\} (5 valeurs ne sont pas représentées du
  tout tandis qu'une valeur est représentée neuf fois).

  On peut considérer cette table comme les coefficients d'un polynôme, dont il
  convient d'ignorer les deux premiers termes. La main du premier exemple
  correspond ainsi à $2\times X^2+X^5$ (on ne retient que la présence de deux
  paires et d'un groupe de cinq valeurs identiques) tandis que le full est
  $X^9$.

  \bigskip Nous voila prêts à modifier la fonction \texttt{affiche()} pour
  donner une forme textuelle de la combinaison obtenue. On utilisera les
  abréviations grecques ou latines pour les valeurs supérieures à 4. La main
  précédente est ``2 paires, 1 penta'' tandis que $X^4 + X^6$ est ``1 carré, 1
  hexa'' et le full ``1 nona''.

  Pour la valeur de la combinaison, on évaluera le polynôme pour $X=2$
  (attention, 4 paires sont plus fortes qu'un triple) ou $X=10$ (attention aux
  débordements de capacité, on peut obtenir des valeurs de main négatives si on
  s'y prend mal).

  \noindent
  \begin{minipage}{.58\linewidth}
    \Question Faites en sorte de permettre à l'utilisateur de demander à
    relancer des dés (comme on redemande des cartes au Poker) en indiquant 5
    booléens après chaque lancé. Dans l'exemple suivant, les caractères
    soulignés sont tapés par l'utilisateur qui demande à relancer les 3 derniers
    dés.
  \end{minipage}\hfill\begin{minipage}{.4\linewidth}
    \begin{Verbatim}[numbers=none,commandchars=\\\{\}]
Moi : 3 5 4 4 5 (deux paires)
Vous: 3 3 6 1 2 (paire)
Que voulez vous faire? \underline{0 0 1 1 1}
Vous: 3 3 3 4 4 (house)
VOUS GAGNEZ !
    \end{Verbatim}
  \end{minipage}

  \Question Limitez les cas d'égalité en faisant en sorte qu'entre deux Fulls,
  le gagnant soit celui dont les dés marquent la plus grande valeur. Faites de
  même pour les combinaisons ``N d'un coup''. Pour départager deux ``House'',
  on regarde d'abord la valeur des triplettes avant de regarder la valeur des
  paires si les triplettes sont équivalentes. On départage les ``Deux paires''
  de façon similaire. Un ``Six Straight'' (2-6) est plus fort qu'un ``Five
  Straight'' (1-5).

  \Question \textit{(mini projet)} Dotez l'ordinateur d'une stratégie pour lui
  aussi relancer certain de ses dés. Cette décision sera prise par le biais
  d'une fonction \run{int *rejoue(int *hand, int hand\_size)}. 

  \Question Implémentez plusieurs fonctions de ce type (tout rejouer, rien
  rejouer, rejouer un dé une fois sur deux, ainsi que des ``vraies''
  stratégies), et faites jouer les AI les unes contre les autres sur un grand
  nombre de parties pour déterminer la meilleure approche. Faites jouer votre
  AI contre celle de votre voisin.
\end{Exercice}


\end{document}

%%% Local Variables:
%%% coding: utf-8
