\documentclass[10pt]{article}\usepackage[correction]{esial}
\CSH\unA\sloppy

\usepackage[utf8]{inputenc}
\hypersetup{colorlinks=false,pdfborder={0 0 0}}

\begin{document}
\title{TP1: Premiers programmes C}
\maketitle


\Exercice \textbf{Premier programme, et entrées/sorties.}
\Question Écrivez un programme C qui calcule le maximum de 3 entiers saisis au
clavier par l'utilisateur, et qui affiche le résultat. Compilez-le sous le nom
\file{max3}. 


\Question Vérifiez que votre programme est correct. Quels tests effectuez--vous
pour cela ?


\Exercice \textbf{Encore des entrées/sorties: problèmes de date.}
Les signes astrologiques sont, en fonction de la date, les suivants: 


\begin{minipage}{.3\linewidth}
  \begin{itemize}
  \item Bélier: 21/3 au 19/4
  \item Taureau: 20/4 au 20/5
  \item Gémeaux: 21/5 au 20/6
  \item Cancer: 21/6 au 21/7
  \end{itemize}
\end{minipage}\hfill\begin{minipage}{.3\linewidth}
  \begin{itemize}
  \item Lion: 22/7 au 22/8
  \item Vierge: 23/8 au 22/9
  \item Balance: 23/9 au 22/10
  \item Scorpion: 23/10 au 21/11
  \end{itemize}
\end{minipage}\hfill\begin{minipage}{.3\linewidth}
  \begin{itemize}
  \item Sagitaire: 22/11 au 21/12
  \item Capricorne: 22/12 au 19/1
  \item Verseau: 20/1 au 18/2
  \item Poisson: 19/2 au 20/3
  \end{itemize}
\end{minipage}

\Question Écrivez un programme demandant la date de naissance de l'utilisateur
et indique son signe astrologique. Si les valeurs entrées pour le jour ou le
mois ne sont pas valides, afficher un message d'erreur et quiter le programme.

\Question Indiquez le jour de la semaine à laquelle l'utilisateur est né, en
utilisant la formule de Zeller: Le jour de la semaine de la date
$\Delta=(j,m,a)$ est 

$$w=j+\left\lfloor 2.6\times t - 0.2 \right\rfloor + d + 
 \left\lfloor \frac{d}{4} \right\rfloor + 
 \left\lfloor \frac{c}{4} \right\rfloor + 5\times c
$$

Avec $\displaystyle t=\left\{\begin{array}{ll}\\
    m+10&\mbox{ (si } m\leq 2)\\
    m-2&\mbox{ (sinon)}
  \end{array}\right.$;  
  $\displaystyle
  b=\left\{\begin{array}{ll}\\
    a-1&\textrm{ (si } m\leq 2)\\
    a&\textrm{ (sinon)}
  \end{array}\right.$; 
  $\displaystyle
  c=\left\lfloor\frac{b}{100}\right\rfloor$;
  $\displaystyle
  d=b-100\times c$.
  
Ensuite, si $w\equiv 0\textrm{ (mod }7)$, $\Delta$ est un lundi. Si cette
grandeur vaut 1, $\Delta$ est un mardi. Pour 2, c'est un mercredi, etc.

\Question Demandez également son nom à l'utilisateur pour améliorer
l'esthétique des affichages.

\bigskip\Exercice\textbf{Manipuler des entiers: le triangle de Pascal}
%----------------------------------------------

On souhaite écrire un programme qui construit le triangle de Pascal de
degré N et l'affiche jusqu'à la diagonale incluse. \\

Les éléments du triangle de Pascal se calculent comme suit (si P est
la matrice mémorisant le triangle)\,: \\
$P_{i,j} = P_{i-1,j} + P_{i-1,j-1}$

Voici quelques exemples d'affichage pour différentes valeurs de N.

\bigskip\noindent\begin{minipage}{.45\linewidth}
  \begin{Verbatim}[label=Affichage quand N vaut 6]
1 
1  1
1  2  1
1  3  3  1
1  4  6  4  1
1  5  10 10 5  1
1  6  15 20 15 6  1    
  \end{Verbatim}
\end{minipage}\hfill
\begin{minipage}{.45\linewidth}
  \begin{Verbatim}[label=Affichage quand N vaut 7]
1 
1  1
1  2  1
1  3  3  1
1  4  6  4  1
1  5  10 10 5  1
1  6  15 20 15 6  1    
1  7  21 35 35 21 7  1
  \end{Verbatim}
\end{minipage}

\Question Écrivez ce programme. Dans un premier temps, vous écrirez simplement
la fonction \framebox{\texttt{int pascal(int i, int j)}} et vous calculerez
chacune des valeurs de cette façon.

L'approche choisie est relativement inefficace: Afficher le triangle de taille
100 est par exemple très très lent. Le problème vient du fait que de nombreux
calculs sont effectués plusieurs fois. Par exemple, calculer $P_{5,3}$ demande
de calculer $P_{3,2}$ à deux reprises (une fois pour avoir $P_{4,2}$ et une
fois pour avoir $P_{4,3}$). Pour résoudre ce problème, nous allons utiliser un
tableau pour stocker les calculs déjà faits afin d'éviter de recalculer deux
fois la même chose.

\Question Ajoutez un tableau d'entiers en global à votre programme en écrivant
les lignes suivantes en dehors de toute fonction. Elles créent un tableau à
deux entrées de taille $100\times100$, et on peut augmenter la taille en
changeant la valeur de la macro.

\begin{Verbatim}
#define MAX 100
int tab[MAX][MAX];
\end{Verbatim}

Ajoutez une double boucle au début de votre fonction \texttt{main()} pour
remplir le tableau de 0.

\Question Modifiez votre fonction \texttt{pascal()}. Avant de faire le moindre
calcul, vous consulterez la case correspondante du tableau. Si cette case
contient une valeur non-nulle, la fonction retournera directement cette valeur
sans la recalculer. Si la case contient 0, la fonction calculera le résultat
comme avant, et stockera le résultat dans la case avant de le retourner en
résultat. Comparez la vitesse d'affichage du triangle de hauteur 100.

\Question (optionnelle) Il est possible d'optimiser la quantité de mémoire
nécessaire en ne stockant qu'une seule ligne du tableau. Il faut alors faire
les calculs dans le bon ordre pour ne pas effacer des cases dont des calculs
futurs auront besoin. Calculer de droite à gauche permet de faire ceci.

\bigskip\Exercice\textbf{Tableaux de caractères: Cadavres exquis}
%----------------------------------------------
L'objectif de cet exercice est de constituer un écrivain automatique. Pour
cela, il faut initialiser plusieurs tableaux de chaines de caractères: un
tableau de sujets accompagnés d'un article, un tableau de verbes conjugués à la
troisième personne, un tableau de compléments d'objets directs au masculin et
un tableau d'adjectifs au masculin.

\medskip\noindent\begin{minipage}{.6\linewidth}
  \begin{tabular}{|c|c|c|c|c|}\hline
    \textit{Sujet}&\textit{Adjectif}&\textit{Verbe}&\textit{Complément}&\textit{Adjectif}
    \\\hline
    Le cadavre&exquis&boira&le vin&nouveau\\\hline
    Le chat&noir&mange&un rat&gris\\\hline
    Le chien&pouilleux&aime&un os&moelleux\\\hline
    La grenouille&gluante&capture&un moustique&agressif\\\hline  
  \end{tabular}
\end{minipage}\hfill\begin{minipage}{.35\linewidth}
  Ce jeu fut inventé par les surréalistes vers 1954. La première
  pharse construite sur ce modèle fut «Le cadavre exquis boira le vin nouveau»,
  ce qui donna le nom du jeu.
\end{minipage}

\Question Le premier problème a résoudre est de savoir comment stocker les
morceaux de phrases qui seront combinés ensuite. Le plus simple est d'utiliser
des tableaux statiques de chaines statiques. Vous avez deux possibilités,
présentées dans les deux blocs suivants. Choisissez l'une d'entre elles, et
écrivez le début de votre programme. 


\begin{Verbatim}[label=Première approche pour le stockage des données]
char *nom[]={"le cadavre","le chat","le chien","la grenouille"};
char *adj1[]={"exquis","noir","pouilleux","gluant"};
...  
\end{Verbatim}

\begin{Verbatim}[label=Deuxieme approche pour le stockage des données]
char *elements[][]={
  {"le cadavre","le chat","le chien","la grenouille"},
  {"exquis","noir","pouilleux","gluant"},
  ...
  };
\end{Verbatim}

\noindent\begin{minipage}{.52\linewidth}
  \Question Complétez votre programme.\\
  Obtenir un nombre aléatoire en C n'est pas très compliqué. Il faut utiliser
  la fonction \texttt{rand()} qui retourne un entier entre 0 et
  \texttt{RAND\_MAX}. Le problème est que la suite pseudo-aléatoire est
  toujours la même par défaut. Initialiser cette suite à l'heure actuelle est
  une bonne solution à ce manque de variété.

\end{minipage}\hfill
\begin{minipage}{.45\linewidth}
  \begin{Verbatim}[gobble=4,numbers=right]
    #include <stdlib.h>
    #include <time.h>

    int main() {
      // Initialise le pseudo aleatoire
      srand(time(NULL));
      // tire un nombre entre 0 et 3
      int nb = rand() % 4;
    }
  \end{Verbatim}
\end{minipage}


\bigskip\Exercice\textbf{Structures en C: le programme annuaire}

%----------------------------------------------

On souhaite créer un programme en C gérant un annuaire très simplifié qui
associe à un nom de personne un numéro de téléphone.

\Question Créer une structure Personne pouvant contenir ces informations (nom
et téléphone). Le nom peut contenir 32 caractères et le numéro 16
caractères.

\Question Créer une nouvelle structure qui va représenter le carnet d'adresses.
Cette structure Carnet contiendra un tableau de 20 Personne et un compteur
indiquant le nombre de personnes dans le tableau.

\Question Créer ensuite une fonction qui renvoie une structure Personne en
prenant en argument un nom et un téléphone.

\Question Rajouter une fonction qui affiche les informations contenues dans la
structure Personne passée en argument.

\Question Créer une fonction qui ajoute une personne dans un carnet.

\Question Créer une fonction qui affiche un carnet.

\Question À partir des étapes précédentes, faire programme gérant un carnet
d'adresse. Créer un menu qui propose d'ajouter une nouvelle personne,
d'afficher le carnet ou de quitter.

\medskip
Une extension possible serait d'ajouter une fonction de sauvegarde de
l'annuaire dans un fichier, et de faire en sorte que les données sauvegardées
puissent être lues automatiquement au démarrage. 

\Exercice \textbf{Pour aller plus loin: les pyramides en C.}

Reprenez problème des pyramides vu en TD et en TP dans le module de
TOP\footnote{Ces sujets sont disponibles sur la page
  \url{http://www.loria.fr/~quinson/teach-prog.html}}. Implémentez la forme
complète de ce problème, où les numéros des billes ne sont pas forcément des
nombres successifs. Utilisez l'algorithme le plus efficace vu pour ce problème,
c'est-à-dire le backtracking avec remplissage par diagonale (et utilisation de
la fonction \texttt{propage()}).

\Question Quel est le nombre entier $n$ le plus petit tel que l'on puisse
remplir une pyramide de hauteur 6 avec toutes les numéros de billes utilisées
inférieurs à $n$?

\Question Quelle est la taille de pyramide maximale pour laquelle vous parvenez
à trouver un remplissage? 

\medskip\noindent\textit{Si vous parvenez à résoudre cet exercice,
  bravo. Faites un mail à \url{Martin.Quinson@loria,fr} pour montrer votre
  solution.}
\end{document}

