\documentclass[10pt]{article}
\usepackage[utf8]{inputenc}
\usepackage{esial}\CSH\1A
%\usepackage[correction]{esial}\CSH\1A
\newcommand{\cd}[1]{\medskip\noindent\file{\null\hspace{-1em}[#1] }}
\newcommand{\touche}[1]{\hbox{$<$#1$>$}}
\newcommand{\ctrl}[1]{\touche{ctrl-#1}}
\newcommand{\tab}{\touche{TAB}}
\sloppy
\usepackage{textcomp,amstext}

\newcommand{\unix}[1]{\hspace*{2cm}{\tt #1}}
\newcommand{\fich}[1]{{\bf \em #1}}

\newcommand{\BoxRep}{\ifcorrection{\boxtimes}{\Box}}

\title{Devoir surveillé du 13 juin 2007}
\begin{document}
\maketitle
\thispagestyle{empty}

\begin{quote}
  Tous documents interdits. Les exercices sont indépendants. La correction
  tiendra compte de la qualité de la rédaction et de la présentation. Barème
  approximatif.
\end{quote}


\begin{Exercice} 
  On souhaite écrire une fonction qui échange le contenu de deux zones mémoires
  de taille \texttt{x} et d'adresses respectives \texttt{a} et \texttt{b}. Si
  les deux blocs se recouvrent, le comportement de la fonction n'est pas
  défini. Le prototype de la fonction est le suivant : 
  \run{void memex(void *a, void *b, size\_t x);}
  

\Question \textbf{(3 pts)} Écrivez cette fonction.

\begin{Reponse}
  \begin{Verbatim}
void premier_cara(void *a, void *b, size\_t x) {
  int i;
  char c;
  for (i=0; i<x; i++) {
    c=a[i];
    a[i] = b[i];
    b[i] = c;
  }
}
  \end{Verbatim}
\end{Reponse}
\end{Exercice}


\begin{Exercice} Lecture d'une chaîne de caractères \textbf{(7 pts)} 

  Dans cet exercice, nous nous intéressons aux fonctions \texttt{gets} et
  \texttt{fgets}. La fonction \texttt{gets} permet de lire une ligne (tous les
  caractères jusqu'au premier '$\backslash$n' inclus ou bien jusqu'à la fin de
  fichier) sur l'entrée standard (le clavier en général) et place le résultat
  (en y a joutant un '$\backslash$0' final) en mémoire à l'adresse donnée str
  en paramètre.  Cette fonction renvoie NULL si elle n'a lu aucun caractère
  (fin de fichier) et dans ce cas la mémoire pointée par \texttt{str} reste
  inchangée. Dans tous les autres cas, elle renvoie le pointeur qui lui est
  passé en paramètre. Son prototype est le suivant :
  \run{char *gets(char *str) ;}

  La fonction \texttt{fgets} est quasiment identique à \texttt{gets} si ce
  n'est qu'elle lit les caractères depuis un flux passé en paramètre et qu'elle
  lit au plus \texttt{size} moins un caractères. Elle renvoie NULL si elle n'a
  lu aucun caractère et dans ce cas la mémoire pointée par \texttt{str} reste
  inchangée. Son prototype est le suivant:
  \run{char *fgets(char *str, int size, FILE *stream);}

  \Question (2 pts) Peut-on écrire \texttt{fgets} en utilisant
  \texttt{gets}?  \texttt{gets} en utilisant \texttt{fgets} ? Justifiez vos
  réponses soit en expliquant pourquoi c'est impossible, soit en écrivant la
  fonction répondant à la question.

  \Question (1 pt) Pourquoi est-il fortement déconseillé d'utiliser la
  fonction \texttt{gets} bien qu'elle fasse partie de la bibliothèque standard
  du C ?

  \Question (4 pts) Ecrivez la fonction \texttt{fgets} en utilisant la
  fonction \texttt{fgetc} qui permet de lire un caractère sur un flux donné :
  \run{int fgetc(FILE *stream)}.

\end{Exercice}

\begin{Exercice} \textbf{(5pts)}
  Soit un fichier regroupant les notes d'étudiants sous la forme suivante :
  \begin{Verbatim}
bob123 12 16
henri4 7 14
leon43 4 1
  \end{Verbatim}
La première colonne est le login unix de l'étudiant, la seconde est la note en
projet et la troisième est la note à l'examen final. Les colonnes sont séparées
par une seul espace.

\Question (\textonehalf~pt) Écrire un script shell listant tous les étudiants ayant eu 20
en projet.

\Question (\textonehalf~pt) Écrire un script shell donnant la note de projet de l'étudiant
toto42.

\Question (2 pts) Écrire un script shell calculant la moyenne à
l'examen final.\\
\textbf{1 point de bonus} si votre solution calcule les moyennes avec un
chiffre après la virgule en n'utilisant que \texttt{expr} pour faire les
calculs.

\noindent\textit{Indication:} \run{expr 10 / 4} donne le résultat de la divion entière
entre 10 et 4 (c'est-à-dire 2). \run{expr 10 \% 4} donne le reste de cette
division (c'est-à-dire 2 également).

\Question (2 pts) Écrire un script shell calculant la moyenne de chaque
étudiant.% en considérant que les coefficients des deux épreuves sont égaux.

\end{Exercice}

\begin{Exercice} \textbf{(5 pts)} Écrire un script shell \run{menage dir}
  proposant d'effacer tous les fichiers du répertoire \texttt{dir} indiqué dont
  les noms se terminent par \~~ou par .BAK ou .bak. Pour chaque fichier, le
  script doit demander à l'utilisateur la confirmation de la suppression avant
  de l'effacer.

  \Question (2 pts) Écrivez ce script.\\
\noindent\textit{Indication:} \run{read toto} permet de lire une chaine de
caractères du clavier et de la stocker dans la variable shell \texttt{toto}.

  \begin{Question} (2 pts)
    Modifiez votre script pour parcourir récursivement les sous-répertoires
    de \texttt{dir}.
  \end{Question}

  \begin{Question} (1 pt)
    Modifiez votre script pour qu'il ne pose la question qu'une fois après
    avoir présenté la liste de tous les fichiers qui seraient écrasés.
  \end{Question}

\end{Exercice}
\end{document}

% LocalWords:  Makefile Laurel strings
%%% Local Variables:
%%% coding: utf-8
