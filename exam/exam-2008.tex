\documentclass[10pt]{article}
\usepackage[utf8]{inputenc}
\usepackage{url}
%\usepackage{esial}\CSH\1A
\usepackage[correction]{esial}\CSH\1A
\newcommand{\cd}[1]{\medskip\noindent\file{\null\hspace{-1em}[#1] }}
\newcommand{\touche}[1]{\hbox{$<$#1$>$}}
\newcommand{\ctrl}[1]{\touche{ctrl-#1}}
\newcommand{\tab}{\touche{TAB}}
\sloppy
\usepackage{textcomp,amstext}

\newcommand{\unix}[1]{\hspace*{2cm}{\tt #1}}
\newcommand{\fich}[1]{{\bf \em #1}}

\newcommand{\BoxRep}{\ifcorrection{\boxtimes}{\Box}}

\newcommand{\boite}{$\Box$\xspace}
\newcommand{\boiteRep}{$\BoxRep$\xspace}

\title{Devoir surveillé du 19 mai 2008}
\begin{document}
\maketitle
\thispagestyle{empty}

\begin{quote}
  Tous documents interdits. Les exercices sont indépendants. La correction
  tiendra compte de la qualité de la rédaction et de la présentation.
\end{quote}

\begin{Exercice} \textbf{(2 pts)}
Écrire un script shell prenant un nombre en argument et calculant la somme
arithmétique de ce nombre (\textit{i.e.,} 1+2+3+\ldots+n).

\begin{Reponse}
  \begin{Verbatim}
#!/bin/sh
n=$1
somme=$1
while [ $n -gt 1 ]
do
  n=`expr $n - 1`
  somme=`expr $somme + $n`
done
echo "Somme = $somme."
  \end{Verbatim}
\end{Reponse}
\end{Exercice}

\begin{figure}[b]
  \begin{minipage}{.48\linewidth}   
  \begin{Verbatim}[label=Programme de l'exercice 2]
#include <stdio.h>
#include <stdlib.h>
#include <string.h>

struct fiche {
   char *nom;
   char *prenom;
   int age;
};

typedef struct fiche type_fiche;

void main() {   
  type_fiche *fiche;
  fiche->nom = malloc(sizeof(char));
  strcpy(fiche->nom, "Bon");
  strcpy(fiche->prenom, "Jean");
  fiche->age = 3;
  printf("Nom : %s\n", fiche->nom);
  printf("Prenom : %s\n", fiche->prenom);
  printf("Age : %s\n", fiche->age);
}
      \end{Verbatim}
    \end{minipage}\hfill  %
  \begin{minipage}{.48\linewidth}   
  \begin{Verbatim}[label=Script de l'exercice 3]
#!/bin/sh

var2="expr 1 + 1"
a='$2'

pretty_print(){
  echo "Le pere noël est une $var1"
  var1=légende
}

echo est ce que '$1 + $1 = $'"$2 ?"
echo non plutot \$1 + '$1' = \$$1 
echo en effet $a n\'est pas egal à `$var2`

var1=$4
echo Conclusion :
pretty_print
echo "mais est-ce une $var1?"
  \end{Verbatim}
\end{minipage}
\end{figure}

\begin{Exercice} \textbf{(2 pts)} Lorsque l'on exécute le programme ci-dessous,
  on obtient le message d'erreur suivant: \run{Erreur de
    segmentation}. Pourquoi? Corrigez les (nombreux) problèmes.

  \noindent
  \begin{minipage}{.5\linewidth}   
  \end{minipage}

  \begin{Reponse}
    \begin{itemize}
    \item Il faut faire un malloc de fiche en ligne 13
    \item Il faut malloc()er plus de une case en ligne 14:
      \run{malloc(sizeof(char)*200)}
    \item Il faut un malloc sur le champ prenom de la structure
    \item Pour afficher un entier, on utilise \%d, pas \%s
    \item Il manque les free, et il faudrait tester le retour des malloc
    \end{itemize}

    (+0.25 par erreur trouvée)
  \end{Reponse}
\end{Exercice}

\begin{Exercice} \textbf{(2 pts)} Le script ci-dessous est correct et s'exécute
  sans erreur.  Quel affichage obtient-on lorsque on 
  l'exécute de la façon suivante : \run{script 2 3 1 ordure} ?

  \begin{Reponse}
    \begin{Verbatim}
est ce que $1 + $1 = $3 ?
non plutot $1 + $1 = $2
en effet $2 n'est pas egal à 2
Conclusion :
Le pere noël est une ordure
mais est-ce une légende?      
    \end{Verbatim}
    (.5pt par ligne juste, maximisé à 2pts)
  \end{Reponse}
\end{Exercice}

\begin{Exercice}\textbf{(1 pt)}
  Écrivez un petit script shell permettant de remplacer dans un fichier texte
  tous les nombres flottants par leur partie entière. On considère que les
  nombres flottants sont écrits sans exposant (c'est-à-dire qu'on écrit le
  nombre mille sous la forme 1000 et pas 1E3, ou encore qu'un millième s'écrit
  0.0001 et non pas 1E-3).

  \begin{Reponse}
    \begin{Verbatim}
sed 's/\([0-9]\)\.[0-9]+/\1/g'      
    \end{Verbatim}
  \end{Reponse}
\end{Exercice}

\begin{Exercice}\textbf{(4 pts)}
  On suppose que l'on dispose d'un fichier calepin.txt, contenant des noms et
  des numéros de téléphone rangés de la manière suivante :
  \begin{Verbatim}[label=calepin.txt,gobble=4]
    DUPONT Jean,05.61.75.18.47,21/08/1975,jean.dupont@free.fr
    MARTIN Yvonne,02.23.34.45.56,26/02/1977,yvonne.martin@cegetel.fr
    ...
  \end{Verbatim}

  \Question \textbf{(\textonehalf pt)} Écrire une commande restituant le numéro
  de téléphone des SCHOTT.
  \begin{Reponse}
cat calepin.txt $|$ grep -i shott $|$ cut -f2 -d,
  \end{Reponse}
  \Question \textbf{(\textonehalf pt)} Écrire une commande qui
  compte les gens habitant dans le sud-est (numéro en 04)
  \begin{Reponse}
     cat calepin.txt $|$ cut -f3 -d' ' $|$ cut -f1 -d'.' $|$ grep '04' $|$ wc -l
  \end{Reponse}

  \Question \textbf{(3 pts)} Écrire une commande envoyant un message
  éléctronique contenant ``Bon anniversaire'' à toutes les personnes dont c'est
  l'anniversaire aujourd'hui.

  \noindent \textit{Indication}: \run{date +\%d/\%m} affiche la date du jour
  sous la forme: jj/mm
\end{Exercice}

\begin{Exercice}\textbf{(6 pts)} 
La commande head permet (par défaut) d'afficher à l'écran les 10 premières
lignes d'un fichier dont le nom lui est passé en argument. Par exemple,
pour afficher les 10 premières lignes du fichier truc.c: \run{head truc.c}
%
Écrire un programme C ayant le même comportement que cette commande.
 
\begin{Reponse}
  \begin{Verbatim}
#include <stdio.h>
#include <stdlib.h>
#include <string.h>

int main(int argc, char* argv[]) {

   if (argc != 2) {
      fprintf(stderr, "Pas assez d'arguments");
      exit(1);
   }
 
   FILE *in = fopen(argv[1],"r");
   if (in == NULL) {
      fprintf(stderr, "Pas possible d'ouvrir %s",argv[1]);
      exit(2);
   }
      
   int seen = 0;
   while (seen < 10 && !feof(in)) {
      char c;
      fscanf(in,"%c",&c);
      if (c == '\n') 
	seen ++;
      printf("%c",c);
   }
   fclose(in);
   return 0;
}
\end{Verbatim}

\end{Reponse}
\end{Exercice}

\newpage
\lhead{\LARGE NOM:}
\chead{\LARGE PRÉNOM:}

\begin{Exercice} \textbf{QCM (3 pts)} Répondez sur la feuille fournie. Il peut
  y avoir plusieurs cases valides par question; les réponses fausses seront
  pénalisées.


\begin{itemize}
\item[$\bullet$] Le langage C a été inventé chez $\ldots$

  \begin{tabular}{*{4}{p{.2\linewidth}}}
    \boiteRep~ AT\&T &
    \boite~    bell &
    \boite~    IBM &
    \boite~    Sun
  \end{tabular}

\item[$\bullet$] Quel est l'équivalent de \run{chmod u=rw,go=r fich}?\smallskip
  
  \begin{tabular}{*{4}{p{.2\linewidth}}}
    \boite~    chmod 600 fich &
    \boite~    chmod 640 fich &
    \boiteRep~ chmod 644 fich &
    \boite~    chmod 755 fich 
  \end{tabular}

\item[$\bullet$] Pour debugger un script shell, que faut il ajouter au début?

  \begin{tabular}{*{4}{p{.2\linewidth}}}
    \boite~    set -dg &
    \boite~    set -u  &
    \boiteRep~ set -x  &
    \boite~    set -w 
  \end{tabular}

\item[$\bullet$] En cas de succès, que doit retourner la fonction main?

  \begin{tabular}{*{4}{p{.2\linewidth}}}
    \boiteRep~ 0~~   &
    \boite~    1~~   &
    \boite~    -1    &
    \boite~    void  
  \end{tabular}

\item[$\bullet$] Quel caractère termine classiquement les chaînes en mémoire en
  C ?

  \begin{tabular}{*{4}{p{.2\linewidth}}}
    \boite~    '$\backslash$n'   &
    \boiteRep~ '$\backslash$0'   &
    \boite~    '.'               &
    \boite~    '$\backslash$END'
  \end{tabular}

\item[$\bullet$] Que vaut 7/9*9 en C?

  \begin{tabular}{*{4}{p{.2\linewidth}}}
    \boiteRep~ 0       &
    \boite~    0.08642 &
    \boite~    1       &
    \boite~    42       
  \end{tabular}

\item[$\bullet$] Si p est de type short* et de valeur 0x1000, que vaut p+1?

  \begin{tabular}{*{4}{p{.2\linewidth}}}
    \boite~    0x1001  &
    \boiteRep~ 0x1002  &
    \boite~    0x1003  &
    \boite~    0x1004
  \end{tabular}

\item[$\bullet$] En C, a[i] est équivalent à $\ldots$

  \begin{tabular}{*{4}{p{.2\linewidth}}}
    \boite~    \&(a+i)&
    \boite~     a+i   &
    \boiteRep~ *(a+i) &
    \boite~    a[0] + i
  \end{tabular}

\item[$\bullet$] On obtient l'adresse de la variable v avec $\ldots$

  \begin{tabular}{*{4}{p{.2\linewidth}}}
    \boiteRep~ \&v&
    \boite~    *v &
    \boite~    @v &
    \boite~    \#v
  \end{tabular}

\item[$\bullet$] Quelle(s) commande(s) \textbf{ne ramene(nt) pas}
  l'utilisateur toto dans son répertoire principal?

  \begin{tabular}{*{4}{p{.2\linewidth}}}
    \boite~    cd          &
    \boiteRep~ cd \url{/~} &
    \boite~    cd \url{~}  &
    \boite~    cd \url{~toto}
  \end{tabular}

\item[$\bullet$] Quelle(s) commande(s) affiche le contenu des fichiers dont le
  nom commence par le caractère «a», et fini par un point suivi de deux
  caractères suivis d'un chiffre?

  \begin{tabular}{*{4}{p{.2\linewidth}}}
    \boite~    ls a*.??[0-9]    &
    \boite~    ls a*.??\#       &
    \boiteRep~ cat a*.??[0-9]    &
    \boite~    cat a*.??\#       
  \end{tabular}

\item[$\bullet$] Par quel caractère doit commencer une ligne de commande dans
  un Makefile?

  \begin{tabular}{*{4}{p{.2\linewidth}}}
    \boite~    un dollard     &
    \boite~    une dièse      &
    \boite~    une espace     &
    \boiteRep~ une tabulation 
  \end{tabular}

% An array declared as
% int exforsys[100];
% The first element of the array is in
%              exforsys
%              exforsys[1]
%              exforsys[0]
%              None of the Above

\end{itemize}
\begin{Reponse}
  \begin{itemize}
  \item +0.25 par case cochée à juste titre
  \item -0.25 par case cochée à tord
  \item 0 si blanc
  \end{itemize}
\end{Reponse}
\end{Exercice}

\end{document}

\begin{Exercice}\textbf{(3pts)} % Pour chacune des questions suivantes, vous répondrez en une
%  seule ligne de commande.
%   \begin{enumerate}
%   \item Envoyer la liste détaillée des fichiers du répertoire courant à toto@loria.fr
%   \item Idem, mais on ne souhaite pas afficher les informations concernant les
%     répertoires. 
%   \end{enumerate}
  On suppose que l'on dispose d'un fichier calepin.txt, contenant des noms et
  des numéros de téléphone rangés de la manière suivante :
  \begin{Verbatim}[label=calepin.txt,gobble=4]
    DUPONT Jean,05.61.75.18.47,21/08/1975,jean.dupont@free.fr
    MARTIN Yvonne,02.23.34.45.56,26/02/1977,yvonne.martin@cegetel.fr
    ...
  \end{Verbatim}

  \Question \textbf{(\textonehalf pt)} Écrire un script (ou une commande) qui
  effectue la recherche des personnes s'appelant DURAND et qui restitue leur
  addresse éléctronique.

  \begin{Reponse}
cat calepin.txt $|$ grep -i durand $|$ cut -f4 -d,
  \end{Reponse}

%  \Question \textbf{(\textonehalf pt)} Écrire un script (ou une commande) qui
%  compte les gens habitant dans le sud-est (numéro en 04)
%  \begin{Reponse}
%     cat calepin.txt | cut -f3 -d' ' | cut -f1 -d'.' | grep '05' | wc -l
%  \end{Reponse}
  \Question \textbf{(\textonehalf pt)} Écrire un script (ou une commande) qui
  compte les gens ayant 20 ans cette année.
  \begin{Reponse}
    cat calepin.txt $|$ cut -f3 -d, $|$ cut -f3 -d'/' $|$ grep -c 1987
  \end{Reponse}

%  \Question \textbf{(1pt)} Écrire un script (ou une commande) envoyant un
%  message éléctronique contenant ``Bonne Année'' à tout le monde.
  \Question \textbf{(2 pts)} Écrire un script (ou une commande) classant les
  personnes de la plus vieille à la plus jeune.
  \begin{Reponse}
    \begin{Verbatim}
cat calepin.txt | \
  sed 's|^\([^,]*,[^,]*\),\(..\)/\(..\)/\(....\)|\4/\3/\2,\1,\2/\3/\4|' | \
  sort -r | \
  sed 's|^[^,],||'
    \end{Verbatim}
    \begin{itemize}
    \item On écrit les lignes sous la forme:\\
      1975/08/21,DUPONT Jean,05.61.75.18.47,21/08/1975,jean.dupont@free.fr
    \item On les trie
    \item On retire le début qui nous a servi à trier
    \end{itemize}
  \end{Reponse}

%  \Question \textbf{(1pt)} Écrire un script (ou une commande) envoyant un
%  message éléctronique contenant ``Bon anniversaire'' à toutes les personnes
%  dont c'est l'anniversaire.
%
%  \noindent \textit{Indication}: \run{date +\%d/\%m} affiche la date du jour
%  sous la forme: jj/mm

  \noindent \textit{Indication}: \run{sort -r} trie les lignes passées entrée
  standard dans l'ordre décroissant.
\end{Exercice}

%\chead{}
\lhead{\LARGE NOM:}
\chead{\LARGE PRÉNOM:}
\newpage
\begin{Exercice}
  Voici un programme C. Il compile et s'exécute normalement sans erreur.

  \begin{minipage}{.35\linewidth}

\end{Exercice}

      

% \begin{Exercice} \textbf{(3pts)}
%   %
%   L'administrateur d'une machine souhaite être prévenu lorsque le nombre
%   d'utilisateurs d'une machine dépasse une certaine valeur N.  Il souhaite donc
%   écrire un programme qui lui affiche un message chaque fois que cette valeur
%   est dépassée.

%   Pour connaître le nombre courant d'utilisateurs il dispose de la commande
%   \fich{uptime}:

% \begin{Verbatim}
% {administrateur} 27 >uptime
%   9:16am  up 99 day(s), 23:08,  2 users,  load average: 1.47, 1.60, 1.52
% {administrateur} 28 >
% \end{Verbatim}

% \Question \textbf{(\textonehalf pt)} Écrivez un fichier de commande
% \fich{uptime.sh} qui exécute la commande \fich{uptime} toutes les 10 minutes.

% \Question \textbf{(2pt)} Écrivez un programme C qui sera compilé sous le nom
% \fich{uptime\_vers\_users}.  Ce programme prend en argument sur la ligne de
% commande un entier N et une adresse email.  Il lit en permanence au clavier le
% résultat de la commande \fich{uptime} et envoie un couriel lorsque le nombre
% d'utilisateurs dépasse N. 

% \noindent \textit{Indication:} La fonction C \underline{\texttt{int
%     system(char *cmd)}} permet d'exécuter la commande \texttt{cmd} dans un
% shell. On peut utiliser le programme \texttt{mail} dans cette commande.

% \Question \textbf{(\textonehalf pt)} Quelle commande doit exécuter notre
% administrateur pour recevoir un couriel chaque fois qu'il y a plus de 15
% utilisateurs sur la machine ?
% \end{Exercice}

\end{document}



\subsection*{Exercice 6}

L'option {\tt -o} du compilateur {\tt gcc} permet d'indiquer le nom du 
programme exécutable que l'on souhaite générer. Cette option est ``dangereuse''
dans le sens où le compilateur ne fait aucune vérification sur le nom de ce 
programme exécutable. Ainsi la commande suivante efface silencieusement
le fichier source \fich{max2.c}~: \\
	\unix{gcc max2 -o max2.c}

Pour éviter ce genre de problème à un utilisateur débutant, on propose de 
lui écrire une nouvelle commande de nom \fich{compile}, qui accepte les mêmes 
arguments que \fich{gcc}, et qui~: 
\begin{itemize}
\item affiche un message d'avertissement si la commande proposée risque 
	d'effacer un fichier source ;
\item exécute normalement la commande standard \fich{gcc} dans le cas contraire.
\end{itemize}
~ \\

On suppose que cet utilisateur n'écrit et ne compile que des programmes constitués 
d'un seul fichier source. Il utilisera donc la commande \fich{compile} de deux 
manière possible : \\
	\unix{compile -o \mbox{{\em fichier1}} \mbox{{\em fichier2}}} \\
ou \\
	\unix{compile \mbox{{\em fichier1}} -o \mbox{{\em fichier2}}} 


{\bf 1.} 
Ecrivez la commande \fich{compile} sous la forme d'un fichier de commande.

{\bf 2.} 
Ecrivez la commande \fich{compile} sous la forme d'un programme C.


% LocalWords:  Makefile Laurel strings
%%% Local Variables:
%%% coding: utf-8
