\documentclass[10pt]{article}
\usepackage[utf8]{inputenc}
\usepackage[french]{babel}
\usepackage[nu]{esial}
\usepackage{url,subfig,multirow}%csh
\usepackage{moreverb}
\1A\CSH
\begin{document}
\title{Projets de Programmation}
\maketitle

\newcommand{\Para}[1]{\medskip\noindent\textbf{#1}}

\Para{Modalités:} Le projet se fait en binôme.

\Para{Date de fin:} \underline{mi mai 2010}, date à laquelle des
soutenances de projet seront organisées.

\Para{Compte-rendu:}
Avant votre soutenance, vous devrez m'\emph{avoir envoyé} votre rapport au format
PDF ainsi que le code source de votre projet dans une archive tar.gz.
Nommez votre archive et votre rapport avec vos noms de famille,
je n'ai pas envie de me retrouver avec 50 fichiers qui s'appellent {\tt projet.tar.gz}.

\bigskip

Le rapport doit contenir:

\begin{itemize}
\item Une ou deux captures d'écran de votre logiciel (s'il y a lieu).
\item Un résumé court de ce que fait votre logiciel (1 ou 2 paragraphe max).
\item Une mention indiquant si vous êtes d'accord pour une diffusion de
 votre travail, et sous quelle condition: source ok / source non, mais
 binaire ok / rien du tout. Si vous savez ce qu'est une licence
 logicielle, c'est mieux mais pas indispensable.
\item Des indications sur ce qui vous a posé problème et comment vous avez
 relevé ces défis.
\item Un paragraphe expliquant comment vous avez géré votre code source
     (gestion de versions, hébergement).
\item Un décompte du nombre d'heures engagées sur quels points par quel
 membre du binome (conception, codage, tests, rédaction du rapport).
\item Une liste des sites et autres sources d'info que vous avez utilisés.
\end{itemize}

Vous vous attacherez à expliquer brièvement ce que fait votre programme:
nous savons lire du code, il n'est pas nécessaire de détailler l'action de chaque ligne. En
revanche expliciter les structures de données choisies et les modules de votre
programme semble judicieux.

\Para{Archive du projet}

L'archive de votre projet doit contenir

\begin{itemize}
\item un fichier texte nommé INSTALL qui explique comment compiler et exécuter le projet.
\item un fichier texte nommé README qui contient vos noms, une présentation succinte du projet
      (vous pouvez réutiliser une partie de votre rapport), et la mention sur
      votre éventuel accord sur la diffusion du logiciel.
\end{itemize}

\bigskip

Si votre projet a consisté à contribuer à un logiciel libre existant, ne rendez
pas d'archive mais donnez dans votre rapport tous les pointeurs permettant de voir
facilement vos contributions.

\Para{Soutenance:} Les soutenances auront lieu à la mi-mai. Vous aurez 15 minutes
pour défendre votre projet.

\Para{Évaluation:} Les sujets ne sont évidement pas tous de la même difficulté,
et nous en tiendrons compte.

%\Para{Rappel:} La tricherie sera \textbf{sévèrement punie}.  Voir
%\url{http://www.loria.fr/~quinson/teaching.html}

\section*{Projet à réaliser}\vspace{-.5\baselineskip}

Il vous revient de choisir le sujet de votre projet, à condition de respecter les règles suivantes:
\begin{itemize}
\item[$\bullet$] Le projet doit être réalisé sous Linux, et écrit en C (ou à la
  limite en shell).
\item[$\bullet$] Le projet doit être \textbf{original} (pas de mastermind ni de
  gemmified, svp). De plus le même sujet ne peut être pris que par un binôme de
  la promotion. Pour cela, il est indispensable de \textbf{me faire un courriel
    pour réserver votre sujet} dès que vous l'avez choisi
  (\url{martin.quinson@loria.fr}).
\item[$\bullet$] Le projet doit être \textbf{suffisamment complexe} (pas de
  pendu, svp). Dans le cas contraire, votre note s'en ressentira forcément. Si
  vous implémentez un jeu un peu trop simple, il semble indispensable
  d'implémenter une interface graphique et/ou un joueur ordinateur. En cas de
  doutes, faites moi un courriel pour me demander mon avis.
\end{itemize}

\bigskip
Voici quelques pistes possibles pour ceux d'entre vous en panne d'imagination:
\vspace{-\baselineskip}
\section*{Sujets amusants}\vspace{-.5\baselineskip}
L'idée de projet conseillée pour la plupart d'entre vous est de refaire un
petit jeu de logique comme on en trouve des milliers en flash sur internet. Par
exemple, le projet de 2007 était de réimplémenter le jeu bejeweled tel
qu'on le trouve sur \url{www.popcap.com}. Voici un rapide florilège de site
offrant de tels jeux en ligne pour vous inspirer:
\begin{itemize}
\item[$\bullet$] \url{http://www.kongregate.com/}
\item[$\bullet$] \url{http://www.popcap.com/}
\item[$\bullet$] \url{http://www.onlineflashgames.org/}
\item[$\bullet$] \url{http://gamegecko.com/}
\item[$\bullet$] \url{http://www.java4k.com/}
\end{itemize}
Je suis sûr que vous en connaissez des dizaines d'autres, n'hésitez pas à me
donner vos bonnes adresses~ ;)

\vspace{-.5\baselineskip}\section*{Sujets utiles}\vspace{-.5\baselineskip}
Pour ceux qui s'en sentent capables, une autre forme de projets conseillée est
d'aider des logiciels libres. Il s'agira d'identifier un projet existant (par
exemple par le biais de sourceforge ou équivalent), et d'apporter une aide
conséquente à son développement. Il peut s'agir d'implémenter une nouvelle
fonctionnalité, ou de corriger des problèmes existants dans le code.
%
Dans le même genre d'esprit, il est possible de parcourir la liste des
bugs critiques listés dans le projet
Debian\footnote{\url{http://bugs.debian.org/release-critical/}}, et
d'apporter un correctif à plusieurs d'entre eux.

Les règles énoncées plus haut (en C ou shell et suffisamment complexe)
s'appliquent évidement également à cette catégorie de projets.

\vspace{-.5\baselineskip}\section*{Sujets originaux}\vspace{-.5\baselineskip}

N'hésitez pas à faire preuve d'imagination dans votre recherche de
sujet. 

\medskip
\centerline{\textit{Quel que soit votre choix, faites-moi un courriel
pour que je valide votre sujet (ou non).}}

\end{document}

%%% Local Variables:
%%% coding: utf-8
